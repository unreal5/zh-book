\documentclass[math,code,12pt]{amznotes}
\xeCJKsetup{CJKmath=true}
%\usepackage{minted}
\setmintedinline[c++]{breaklines,breakanywhere,bgcolor=none}
\newcommand{\il}[1]{\mintinline{c++}{#1}}%

\newtcbox{\mybox}[1][red]{on line,colupper=black,fonttitle=\bfseries,
 arc=2pt,outer arc=2pt,colback=red!10!white,colframe=#1!50!black,
 boxsep=0pt,left=2pt,right=2pt,top=2pt,bottom=2pt,
 boxrule=0pt,bottomrule=0pt,toprule=0pt}
 
\begin{document}
		%\frontmatter
    %\tableofcontents

\mainmatter\setchapterimage{./images/background}

\chapter{技术}	
	本章呈现许多贯穿本书的C++技术。为了在各式各样的情境(context)中都有用,它们倾向于泛化(一般化,general)和可复用(reusable),如此便可在其他情境中找出它们的应用。有些技术如partial template specialization(模板偏特化)是语言本身的特性,有些如``编译期assertions'' 则需藉由代码实作出来。	
	
	本章之中你将了解下列这些技术和工具:
	\begin{itemize}
		\item Partial template specialization(模板偏特化)
		\item Local classes(局部类)
		\item 型别和数值之间的映射(Int2Type和Type2Type class templates)
	\end{itemize}
\end{document}
