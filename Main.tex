\documentclass[math,code,12pt]{amznotes}
\xeCJKsetup{CJKmath=true}
%\usepackage{minted}
\setmintedinline[c++]{breaklines,breakanywhere,bgcolor=none}
\newcommand{\il}[1]{\mintinline{c++}{#1}}%

\newtcbox{\mybox}[1][red]{on line,colupper=black,fonttitle=\bfseries,
 arc=2pt,outer arc=2pt,colback=red!10!white,colframe=#1!50!black,
 boxsep=0pt,left=2pt,right=2pt,top=2pt,bottom=2pt,
 boxrule=0pt,bottomrule=0pt,toprule=0pt}

%%%%%%%%%%%%%%%%%%%%%%%%%%%%%%%%%%
\usepackage{fontawesome5}
%\hspace*{\fill} #1 \hspace*{\fill}\faRemove\\文件(F)\quad 动作(A)\quad 编辑(E)\quad 查看(V)\quad 帮助(H)\hspace*{\fill}
\newtcolorbox{console}{colback=black,
    colupper=white,
    fontupper=\small\ttfamily\bfseries,
    fonttitle=\small\sffamily,
    colback=amzcodeboxcolor,
    frame style={opacity=0.25},
    colframe=amzcodeboxcolor,
    center title,
    left = 2pt,
    width=\linewidth,
    listing only,
    drop shadow,
    right = 2pt,
    top = 1mm,
    boxsep =1pt,
    arc=2pt,
    title ={\faTerminal\hspace*{\fill}控制台\ \faBatteryThreeQuarters},
    %listing options={style=tcblatex,language=sh},
    } 
\begin{document}
		%\frontmatter
    %\tableofcontents

		\mainmatter\setchapterimage{./images/background}
		\chapter{世界空间 vs Rig空间}
		Rig空间和世界空间之间相互转换可以使用\il{FromWorld}和\il{ToWorld}。\dfntxt{Rig空间}在control rig中也称\dfntxt{Global Space},这种说法容易让人混淆,要注意。牢记:$Rig Space \iff Global Space$。		\dfntxt{Rig空间}是相对角色的骨架的坐标。
		
		在control rig蓝图中可以使用对应的结点,但对于c++,可以使用\il{ExecuteContext}的成员函数:
		\begin{amzcode}{c++}
		FTransform GlobalTM = ExecuteContext.Hierarchy->GetGlobalTransform(BoneKey);
		FTransform WorldTM = ExecuteContext.ToWorldSpace(GlobalTM);
		\end{amzcode}
		
		动画每帧都会调用\mybox{Forwards Solve}结点,但我们只想调用一次,并且在所有的事件发生前调用,此时可以使用\mybox{Construction Event}。
		
		对于需要启动control rig控制台绘制的,需要打开以下控制台命令。
		\begin{console}
		a.AnimNode.ControlRig.Debug 1
		
		ControlRig.EnableDrawInterfaceInGame 1
		\end{console}
		\section{Fullbody IK}
		\mybox{Full Body IK} 节点它能让我们为角色的某条肢体指定一个要去对齐的目标点,然后它会尝试求解\dfntxt{整个身体}(\emph{fullbody}),以到达那个目标点。
		
		\begin{tcolorbox}[breakable,enhanced,bicolor,sidebyside,lefthand width=.45\linewidth,
		 sharp corners,boxrule=.4pt,colback=green!5,colbacklower=green!50!black!50,nobeforeafter]
		 \begin{enumerate}
		 \item Root:通常设置为FullBodyIK解算时\emph{最顶层}(top-most)的骨骼,Root之上的骨骼链通常忽略,不参与解算。
		 \item Effectors:设置移动的末端骨骼的数组。对于蝎子示例,有6个末端骨骼,添加6个数组成员。
		 \end{enumerate}
		 \tcblower
		 \includegraphics[width=\linewidth]{./images/fullBodyIK}
		 \end{tcolorbox}
		 \subsection{FTransform}
		 \begin{enumerate}
		 \item 由缩放(Scale)、旋转(以四元数表示)和位移(Translation)组成的变换。
		 \item  变换可用于在不同空间之间进行转换,例如将位置和方向从局部空间转换到世界空间。
		 \item  对位置向量的变换按顺序应用:Scale -> Rotate -> Translate。
		 \item  对方向向量的变换按顺序应用:Scale -> Rotate。
		 \item  在复合变换时,顺序很重要:$C = A \times B$ 得到的变换 C,在对后续对象进行变换时,逻辑上等价于先应用 A,再应用 B。注意,这与四元数(\il{TQuat<T>})相乘的顺序相反。
		 \item  示例:\il{LocalToWorld = (DeltaRotation * LocalToWorld)} 将在局部空间中按 DeltaRotation 改变旋转。
		 \item  示例:\il{LocalToWorld = (LocalToWorld * DeltaRotation)} 将在世界空间中按 DeltaRotation 改变旋转。
		 \end{enumerate}
\begin{genbox}{Transform的乘法计算}
在 UE 的 TTransform 中,$C = A * B$ 表示:随后对点调用 \il{C.TransformPosition(P)} 时,等价于 \il{B.TransformPosition(A.TransformPosition(P))},即先应用 $A$,再应用 $B$(注意:四元数相乘顺序与“应用顺序”相反)。

骨骼层级中,Local 是子相对父的局部变换(Child→Parent),ParentWorld 是父到世界(Parent→World)。要得到子到世界(Child→World),应先 Local 再 ParentWorld,因此 $ChildWorld = Local * ParentWorld$。

为何非交换:由于旋转和平移的耦合,$A * B$ 通常不等于 $B * A$,顺序改变会得到不同的结果。
\end{genbox}		 


\subsection*{核心目标}
为了计算速度,需要先得到角色自上一帧以来在世界空间(World Space)的位移。为此:
\begin{itemize}
\item 记录上一帧的世界变换(PreviousWorldTransform)。
\item 在当前帧计算出从上一帧到当前帧的世界空间差值(WorldDelta)。
\item 将上一帧的世界变换在每次函数调用末尾更新为当前值,确保下一帧对比的是“真正的上一帧”。
\end{itemize}

\subsection*{变量与类型}
\begin{itemize}
\item \textbf{PreviousWorldTransform}(类型:Transform)
\begin{itemize}
\item 初名 \texttt{PreviousWorldLocation},后改为 \texttt{PreviousWorldTransform},因为 Transform 还包含旋转与缩放(虽然当前不使用,但命名更准确)。
\item 通过 \texttt{Compile} 与 \texttt{Save} 生效。
\end{itemize}
\item \textbf{WorldDelta}(类型:Transform)
\begin{itemize}
\item 表示自上次更新以来的世界空间差值(位移/相对变换)。当前主要关注平移分量,但其本质为完整 Transform。
\end{itemize}
\end{itemize}

\subsection*{保存上一帧世界变换(每帧末尾执行)}
\begin{enumerate}
\item 在 \il{CalculateVelocity} 函数中,每次调用(每帧)都需要更新 \textbf{PreviousWorldTransform} 为当前的世界变换。
\item 获取当前 Rig 的世界变换:
\begin{itemize}
\item 取 Rig Space 原点(中心,坐标(0,0,0))的 Transform。
\item 使用 \il{To World} 节点将其从 Rig Space 转换到 World Space。
\item 说明:\il{To World} 在未显式输入 Transform 时,默认使用零平移、零旋转(缩放无关),即“无偏移”的 Rig Transform,再将其转换到世界空间。这等价于查询“Rig 在世界中的位置”(其中心点在关卡中的位置)。
\end{itemize}
\item 将转换得到的当前世界变换通过 \il{Set PreviousWorldTransform} 写入变量,但要把这一写入放在图(逻辑)靠右、即\emph{本帧所有计算完成之后},避免本帧计算误用“刚更新的当前值”,而应使用“真正的上一帧值”。
\end{enumerate}

\subsection*{计算与上一帧的差值(WorldDelta)}
因为Transform没有直接支持“\dfntxt{减法}”运算,所以使用了$C=A*B.Inverse()$,相当于做减法。
\begin{enumerate}
\item 再次获取当前帧的世界变换(同上:\il{To World},无输入即默认从 Rig 原点出发)。
\item 取 \textbf{PreviousWorldTransform} 的 \il{Inverse}(求逆)。
\item 使用 Transform 乘法 \il{Multiply (Transform)} 计算差值:
\begin{itemize}
\item 重要:Transform 乘法是\textbf{非交换}的(non-commutative),顺序必须严格为:
\[
\text{CurrentWorld} \times \text{Inverse}(\text{PreviousWorld})
\]
\item 其中 \textbf{CurrentWorld} 由 “Rig 原点 (\rightarrow) \il{To World}” 得到;\textbf{PreviousWorld} 即上一帧保存的世界变换。
\end{itemize}
\item 在更新 \textbf{PreviousWorldTransform} 之前,先将上述乘法结果保存到新变量 \textbf{WorldDelta}(通过 \il{Set WorldDelta})。
\item \il{Compile} 与 \il{Save} 确认更改。
\end{enumerate}

\subsection*{公式化表达}
令
$T_{\text{rig},0} \text{ 为 Rig 空间的零偏移变换(原点)},\\
W_{\text{curr}} = \texttt{ToWorld}(T_{\text{rig},0}),\\\
W_{\text{prev}} = \text{PreviousWorldTransform}.
$\\
则本帧与上一帧的世界空间相对变换(差值)为$\text{WorldDelta} = W_{\text{curr}} \times W_{\text{prev}}^{-1}$.
当前关注其平移长度可作为“位移量”,在结合时间后可用于速度计算。

\subsection*{关键注意事项与节点}
\begin{itemize}
\item \textbf{顺序敏感}:Transform 乘法非交换,必须用“当前 (\times) 上一帧的逆”。
\item \textbf{更新时机}:\textbf{PreviousWorldTransform} 的更新放在\emph{本帧计算之后},确保用于对比的是“真正的上一帧值”。
\item \textbf{空间概念}:Rig Space 的原点经过 \il{To World} 得到 Rig 在关卡(World Space)中的位置。
\item \textbf{默认输入}:\il{To World} 无输入时采用零变换(平移与旋转为零,缩放无关),表示“无偏移”的 Rig Transform。
\item \textbf{使用的节点/操作}:
\begin{itemize}
\item \il{To World}(Rig (\rightarrow) World 转换)
\item \il{Inverse}(求逆变换)
\item \il{Multiply (Transform)}(变换乘法)
\item \il{Set PreviousWorldTransform},\il{Set WorldDelta}
\end{itemize}
\end{itemize}

\subsection*{阶段性结果与后续}
\begin{itemize}
\item 已完成:每帧保存上一帧的世界变换,并计算 \textbf{WorldDelta},用于表示“自上次更新以来的世界空间差值”。
\item 下一步(预告):将在后续章节测试并验证该差值是否正确反映角色在世界中的位移。
\end{itemize}



		\setchapterimage{./images/j20}
		%\setchapterimage{}
    \chapter{过程性动画}
    Control rig 使用c++实现步骤:
    \section{球形追踪}
    在程序化动画中,角色的脚部会直接移动到其原始动画位置,这会导致脚部穿过地板或其他障碍物,看起来不真实。
    
    \subsection{解决方案:使用球体追踪(Sphere Trace)}
    为了解决脚部穿透问题,可以使用\textbf{球体追踪}的方法来动态调整脚部的位置,使其能够正确地与地面或障碍物进行交互。
    
    \begin{tecbox}{实现步骤}{tec1}
    \begin{enumerate}
        \item \textbf{创建球体追踪节点}:在动画蓝图中,使用“Sphere Trace by Channel”节点。这个节点会从一个起始点向一个结束点发射一个带半径的球体,检测路径上是否与其他物体发生碰撞。
    
        \item \textbf{设置追踪的起点和终点}:
        \begin{itemize}
            \item \textbf{终点 (End Point)}:设置为脚部IK的原始目标位置(即动画中脚应该在的位置)。
            \item \textbf{起点 (Start Point)}:在终点位置的基础上,增加Z轴的高度(例如50个单位),这样追踪射线就是从脚的上方垂直向下发射。
        \end{itemize}
    
        \item \textbf{处理追踪结果}:
        \begin{itemize}
            \item 球体追踪会返回一个布尔值(是否击中)和一个“击中位置”(\il{Hit Location})。
            \item 如果追踪击中了物体(如地面),则将脚的IK目标位置设置为这个“击中位置”。
        \end{itemize}
    \end{enumerate}
    \end{tecbox}
    \begin{genbox}{遇到的问题与优化}
    \begin{enumerate}
        \item \textbf{问题一:脚在空中时位置错误}
        \begin{itemize}
            \item \textbf{现象}:当角色跳起,脚在空中时,球体追踪没有击中任何物体。此时,“击中位置”会返回一个零向量(0,0,0),导致双脚在空中时瞬间吸附到世界坐标原点。
            \item \textbf{解决方案}:使用一个\textbf{IF条件判断节点}。
            \begin{itemize}
                \item 如果球体追踪的返回值为\textbf{True}(击中物体),则使用“击中位置”作为脚的IK目标。
                \item 如果返回值为\textbf{False}(未击中),则继续使用原始的动画目标位置。
            \end{itemize}
        \end{itemize}
    
        \item \textbf{问题二:脚部陷入地面}
        \begin{itemize}
            \item \textbf{现象}:即使追踪击中了地面,脚看起来仍然部分陷入地面。这是因为骨骼的轴心点(Pivot)在脚踝,而不是在脚底。将脚踝位置设置为地面高度,自然会导致脚掌部分在地面以下。
            \item \textbf{解决方案}:在“击中位置”的基础上,给Z轴增加一个固定的偏移量(例如15个单位),然后再将其设置为IK目标。这个偏移量相当于从脚踝到脚底的高度,从而将脚正确地抬高到地面上。
        \end{itemize}
    
        \item \textbf{问题三:状态切换时的抖动}
        \begin{itemize}
            \item \textbf{现象}:当脚从空中接触地面时,由于Z轴偏移只在“击中”状态下应用,可能会产生一个微小的“吸附”或“抖动”效果。
            \item \textbf{解决方案}:将Z轴的高度偏移应用到\textbf{所有情况}下(无论是否击中物体)。这样可以确保脚部在空中和在地面的高度基准一致,消除状态切换时的不连贯。
        \end{itemize}
    \end{enumerate}
    \end{genbox}
    \subsection{测试与结论}
    视频中通过在场景中添加一个立方体并设置其碰撞属性,来测试脚部IK是否能正确地踩在新的障碍物上。实验证明该方法有效,可以实现单脚或双脚与不同高度的物体进行正确的交互。
    
    然而,对于视频作者后续的纯程序化动画(不依赖传统动画文件)目标而言,这个基于追踪的IK修正系统并非必需。因此,在演示了其原理和实现方法后,作者最终删除了这部分逻辑,恢复到更简单的IK设置。
\begin{codebox}{MakeBasicIKParams}{MakeBasicIKParams}
	\begin{amzcode}{c++}
USTRUCT(meta = (DisplayName = "MakeBasicIKParams"))
struct PROCEDURALANIMCOURSE_API FRigUnit_MakeBasicIKParams : public FRigUnit_DynamicHierarchyBase
{
	GENERATED_BODY()
	FRigUnit_MakeBasicIKParams(){}
	
	RIGVM_METHOD()
	void Execute() override;

	UPROPERTY(meta = (Input))
	FRigElementKey FootKey;

	UPROPERTY(meta = (Output))
	FRigElementKey ItemA;

	UPROPERTY(meta = (Output))
	FRigElementKey ItemB;

	UPROPERTY(meta = (Output))
	FVector PrimaryAxis;

	UPROPERTY(meta = (Output))
	FVector SecondaryAxis;
};	
  \end{amzcode}
\end{codebox}

\begin{codebox}{实现}{MakeBasicIKParamscpp}
	\begin{amzcode}{c++}
void FRigUnit_MakeBasicIKParams::StaticExecute(const FControlRigExecuteContext& ExecuteContext,
                                               const FRigElementKey& FootKey,
                                               FRigElementKey& ItemA,
                                               FRigElementKey& ItemB,
                                               FVector& PrimaryAxis,
                                               FVector& SecondaryAxis)
{
	if (ExecuteContext.Hierarchy == nullptr) return;

	ItemB = ExecuteContext.Hierarchy->GetDefaultParent(FootKey);
	ItemA = ExecuteContext.Hierarchy->GetDefaultParent(ItemB);
	PrimaryAxis = FVector{-1.f, 0.f, 0.f};
	SecondaryAxis = FVector{0.f, 1.f, 0.f};
	FName FootName = FootKey.Name;
	if (FootName.ToString().Contains("_l")){}
	else
	{
		PrimaryAxis *= -1.f;
		SecondaryAxis *= -1.f;
	}
}	
  \end{amzcode}
\end{codebox}

\section{围绕定点旋转}
接下来我们要处理的是脚的旋转。当脚向后移动时,我们会让它围绕脚掌旋转。当它向后移动得更远时,我们会让它围绕脚尖旋转,这样脚在向后移动时就会抬起来。同样,当它向前移动时,我们会让它围绕脚后跟旋转,在它向前伸展的时候。所有这些都依赖于我们能够围绕另一个点旋转一个点。例如,脚骨的实际位置是在脚踝处。
但我们需要能够围绕脚掌、脚尖或脚后跟来旋转它。所以我们需要一个函数,可以指定一个变换,比如脚骨的位置,然后再指定第二个变换作为旋转中心,比如脚掌、脚尖或者脚后跟,然后基于那个原点旋转该点。要做到这一点,我们将通过点击加号创建一个新函数并将其命名为 \dfntxt{rotateAroundPoint}。

双击它,你应该就进入了函数内部
你会看到入口和返回节点。如果你点击入口节点,你会在右侧看到输入和输出。所以我们要添加一个新的输入,点击加号
并将其设置为一个变换
我们称之为 transformToRotate。在这个例子中,transformToRotate 将是骨骼的变换
它位于脚的脚踝位置。然后,在它被修改后,我们想要输出修改后的版本。所以返回节点将需要一个输出变换。所以在输出上,你可以点击加号
并再次将其设置为一个变换
我们将其重命名为 modifiedTransform。所以基本上我们要做的是
从 transformToRotate 我们将修改它,然后它将进入输出。但我们不会马上连接它,因为我们还没有真正进行修改。接下来,我们需要指定我们想要围绕旋转的点。例如,脚掌或脚后跟。我们将添加一个新的输入,这只是一个向量。我们称之为 pointToRotateAround。或者你可以给它起个更好的名字,但这只是为了方便记忆。所以这将是我们围绕旋转上述变换的点。然后我们需要一个表示旋转量的变量。所以创建一个新变量,并搜索 quot,代表四元数。$我们称之为 rotationAmount$。所以这表示我们将围绕 pointToRotate 旋转原始变换 transformToRotate 的量。然后我们将输出到 modifiedTransform。所以我们要做的第一件事是,从 rotation amount 拖出,然后做乘法。这将是来自 rotation amount 的四元数乘法。那需要被插入到 A。然后从 transformToRotate 下拉,并将其从 rotation 插入到 B。必须是这个顺序,所以 A 和 B 不能交换。所以 A 必须是旋转的量,B 必须是你正在旋转的东西,也就是原始的变换。所以这样做就是用一个旋转另一个。所以现在我们可以把它插入到旋转输出中,因为它被旋转量旋转了。这对旋转有效,所以我们改变了旋转。所以如果例如旋转量是但我们还需要旋转平移。因为例如,在这种情况下,如果我们试图围绕脚掌旋转位于脚踝的脚骨,并顺时针旋转它。脚的位置将会移动,它会围绕 pointToRotateAround 呈弧形旋转。要做到这一点,我们想要得到 transformToRotate 的平移(也就是脚的位置)与 pointToRotateAround 之间的差值。以及该差值与 pointToRotateAround 之间的差值。所以我们创建那个向量,它指向从旋转点到我们正在旋转的骨骼的方向。要做到这一点,我们从 transformToRotate 的平移中拖出,然后做减法。然后减去 pointToRotateAround。然后我们要做的是旋转这个向量。所以我们得到了这条线,它基本上从,比如说,脚尖或脚掌指向脚踝。然后我们想根据旋转量来旋转那个向量。所以我们从结果中拖出,然后搜索 RotateVector。我们将用来旋转它的变换将是旋转量。这就给了我们一个向量的旋转版本。现在如果我们只是简单地将结果插入到平移中,这是不正确的,因为我们从中减去了一个数字。我们减去了 pointToRotateAround,然后旋转了它。所以我们需要做的基本上是把 pointToRotateAround 加回去。所以我们减去了它,所以我们只得到了它自己,那个向量。进行了旋转,然后我们需要把它加回去。所以从结果中拖出,然后做加法。把它插入 A 或 B,对于加法节点来说,顺序无关紧要。然后插入 pointToRotateAround,然后将结果插入平移。现在我们只需要把缩放插入到缩放中,我们不修改缩放,所以没关系。我们可以直接把它插进去,然后编译。
\subsection{四元数}
四元数,可表示绕三维空间中某一轴的旋转。$X、Y、Z、W$ 分量也可表示为\dfntxt{轴/角}格式。组合四元数时顺序很重要:$C = A * B$ 会得到四元数 C,逻辑上先应用 B 然后应用 A 到后续的变换(先右侧再左侧)。注意这与 FTransform 的乘法顺序相反。示例:$LocalToWorld = (LocalToWorld * DeltaRotation)$ 会在局部空间按 $DeltaRotation$ 改变旋转;示例:$LocalToWorld = (DeltaRotation * LocalToWorld)$ 会在世界空间按 $DeltaRotation$ 改变旋转。
\subsection{使用四元数计算旋转}
假定要让空间某点A,围绕另一点P旋转\il{Amount}量,则大致计算如下:
\begin{notebox}
把点 A 平移到以 P 为原点的局部坐标$(A - P)$,用四元数旋转这个向量,然后把结果平移回 P:$Result = P + Amount.RotateVector(A - P)$。
\end{notebox}

\begin{amzcode}{c++}
// 将点 A 围绕枢轴 P 按四元数 Rotation 旋转
static FVector RotatePointAroundPivot(const FVector& A, const FVector& P, const FQuat& Rotation)
{
    return P + Rotation.RotateVector(A - P);
}

// 示例用法:
// 1) 通过轴/角(角度以弧度为单位)
FVector RotateExample_AxisAngle(const FVector& A, const FVector& P, const FVector& Axis, float AngleRad)
{
    FQuat Q = FQuat(Axis.GetSafeNormal(), AngleRad); // AngleRad 为弧度
    return RotatePointAroundPivot(A, P, Q);
}

// 2) 通过欧拉角(以度为单位)
FVector RotateExample_EulerDegrees(const FVector& A, const FVector& P, const FRotator& RotatorDegrees)
{
    FQuat Q = RotatorDegrees.Quaternion(); // FRotator 以度为单位
    return RotatePointAroundPivot(A, P, Q);
}
\end{amzcode}
对于\il{struct FRigUnitMutable : public FRigUnit}相比\il{FRigUnit}多了执行引脚,并且调用的静态函数的签名有所不同。
%\amzindex
\end{document}
\end{document}
