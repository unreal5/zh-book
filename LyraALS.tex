\documentclass[10pt,CJKmath]{zhbook-v1}
%\xeCJKsetup{CJKmath=true}
\usepackage{tikz}
\usetikzlibrary{arrows.meta,shapes.geometric,positioning,shadows,fit,backgrounds}
% Global TikZ styles for consistent, polished diagrams
\tikzset{
  base/.style={font=\small, align=center,inner sep=2pt,outer sep=0mm},
  block/.style={rectangle, draw=black!60, rounded corners, minimum width=20mm, minimum height=8mm, align=center, fill=teal!6, drop shadow, base},
  proc/.style={rectangle, draw=black!60, minimum width=38mm, minimum height=8mm, align=center, fill=orange!6, drop shadow, base},
  box/.style={rectangle, draw=black!45, rounded corners, minimum width=40mm, minimum height=8mm, align=center, fill=gray!6, base},
  arrow/.style={-{Latex[length=3mm]}, very thick, line cap=round, color=black!70}
}

\usepackage{fontawesome5}
\setmintedinline[c++]{breaklines,breakanywhere,bgcolor=none,breakanywheresymbolpre=\rotatebox{-90}{\mbox{\tiny\textcolor{pink}{\faLevelDown*}}}}
\newcommand{\il}[1]{\mintinline{c++}{#1}}%

\newtcbox{\mybox}[1][red]{on line,colupper=black,fonttitle=\bfseries,
 arc=4pt,outer arc=4pt,colback=yellow!10!white,colframe=#1!50!black,
 boxsep=0pt,left=2pt,right=2pt,top=1pt,bottom=1pt,
 boxrule=1pt,leftrule=1pt,rightrule=1pt}
 
 
\begin{document}
		%\frontmatter
    %\tableofcontents

\mainmatter
\pagestyle{fancy}
\setchapterimage{./images/background}

\chapter{启始}
\section{NativeThreadSafeUpdateAnimation VS FAnimInstanceProxy}
对于需要大量计算的动画实例,并希望利用多线程进行求值,推荐使用 \il{FAnimInstanceProxy}。

这两者并非完全互斥的选项,\il{FAnimInstanceProxy} 是实现动画多线程更新的底层结构和推荐模式,而 \il{NativeThreadSafeUpdateAnimation} 是一个更上层的、简化的接口。

\begin{itemize}
	\item 重载 \il{NativeThreadSafeUpdateAnimation}:这是一种更简单直接的方法,适用于逻辑不那么复杂的场景。
			\begin{itemize}
			\item 优点:
			    \begin{itemize}
			    	\item 简单: 只需在 UAnimInstance 子类中重载一个函数,无需创建和管理额外的代理(Proxy)结构体。
			    	\item 快速实现: 对于简单的属性复制和计算,这种方式的代码量更少,更容易理解。
			    \end{itemize}
			
			\item 缺点:
				\begin{itemize}
						\item 功能限制: 它\textbf{本质}上是 \il{FAnimInstanceProxy::Update} 的一个简化包装。所有数据\dfntxt{必须是线程安全}的,并且必须在 NativeUpdateAnimation(游戏线程)中提前准备好,然后在这个函数(工作线程)中使用。
						\item 扩展性差: 随着计算逻辑变得越来越复杂(例如,需要缓存中间结果、跨帧数据等),只使用这个函数会变得混乱。所有逻辑和数据都挤在 UAnimInstance 类中,不符合关注点分离原则。
				\end{itemize}
			\end{itemize}
	\item  使用 \il{FAnimInstanceProxy}:这是UE官方推荐的、更健壮和可扩展的多线程动画更新模式。

	\begin{itemize}
		\item 	优点:
			\begin{itemize}
			\item 结构清晰: 强制将数据和逻辑分离。UAnimInstance 负责从游戏世界(Game Thread)收集数据,\il{FAnimInstanceProxy} 负责在工作线程(Worker Thread)上使用这些数据进行计算。
	    \item 性能更佳: 代理(Proxy)是一个普通的C++结构体(struct),比 UObject 更轻量。在工作线程上传递和操作代理的开销更小。
	    \item 扩展性强: 面对“未来会有大量的求值计算”,\il{FAnimInstanceProxy} 提供了清晰的扩展路径。你可以:
			\item 在 FAnimInstanceProxy::PreUpdate (Game Thread) 中高效地从 UAnimInstance 拷贝数据。
	    \item 在 FAnimInstanceProxy::Update (Worker Thread) 中执行所有复杂的计算。
	    \item 在 FAnimInstanceProxy::PostUpdate (Game Thread) 中将计算结果写回 UAnimInstance(如果需要)。
	    \item 更安全: 这种模式强制你思考线程安全问题,将游戏线程的访问和工作线程的计算明确分开,减少了多线程编程中常见的错误。
			\end{itemize}
	
		\item  缺点:	    更复杂: 需要额外定义一个 \il{FAnimInstanceProxy} 结构体,并重载 CreateProxy 和 DestroyProxy 函数。初始设置的代码量比仅重载 \il{NativeThreadSafeUpdateAnimation} 要多。
	\end{itemize}
\end{itemize}

\subsection{示例}
这是一个基于 ULyraBaseAnimInstance 的简洁示例,展示了如何实现 FAnimInstanceProxy 模式。这个例子遵循了UE官方推荐的最佳实践。

我们将为 ULyraBaseAnimInstance 创建一个代理 FLyraBaseAnimInstanceProxy,并在其中执行多线程计算。

\subsubsection{头文件 (LyraBaseAnimInstance.h)}

在头文件中,我们需要:
\begin{enumerate}
\item 声明代理结构体 FLyraBaseAnimInstanceProxy。
\item  在 ULyraBaseAnimInstance 中重载 CreateProxy 和 DestroyProxy。
\item  添加一些我们希望在工作线程中计算的属性,例如 GroundSpeed。
\item  定义 FLyraBaseAnimInstanceProxy 结构体,它继承自 FAnimInstanceProxy,并包含用于计算的成员变量和函数。
\end{enumerate}

\begin{amzcode}{c++}
// LyraBaseAnimInstance.h

#pragma once

#include "CoreMinimal.h"
#include "Animation/AnimInstance.h"
#include "Animation/AnimInstanceProxy.h" // 必须包含代理头文件
#include "LyraBaseAnimInstance.generated.h"

class ALyraCharacter;
class UCharacterMovementComponent;

// 1. 定义代理结构体,它将包含所有线程安全的数据和求值逻辑
USTRUCT()
struct FLyraBaseAnimInstanceProxy : public FAnimInstanceProxy
{
	GENERATED_BODY()

public:
	// 使用AnimInstance来初始化代理
	FLyraBaseAnimInstanceProxy(UAnimInstance* Instance);

	// 在工作线程上执行的更新函数,所有耗时计算都在这里
	virtual void Update(float DeltaSeconds) override;

	// 在游戏线程上,从AnimInstance拷贝数据到Proxy中
	virtual void PreUpdate(UAnimInstance* InAnimInstance, float DeltaSeconds) override;

public:
	// 在工作线程上计算出的最终值
	UPROPERTY(Transient, BlueprintReadOnly, Category = "Lyra|Proxy Data")
	float GroundSpeed = 0.0f;

	UPROPERTY(Transient, BlueprintReadOnly, Category = "Lyra|Proxy Data")
	bool bIsFalling = false;

protected:
	// 从游戏线程拷贝过来的原始数据
	FVector CharacterVelocity;
	// ... 其他需要从游戏线程拷贝的数据
};


UCLASS(Abstract)
class LYRAALS_API ULyraBaseAnimInstance : public UAnimInstance
{
	GENERATED_BODY()

public:
	ULyraBaseAnimInstance();

	// 在动画蓝图中可以直接使用的属性,由Proxy计算后更新
	UPROPERTY(Transient, BlueprintReadOnly, Category = "Lyra|Movement")
	float GroundSpeed;

	UPROPERTY(Transient, BlueprintReadOnly, Category = "Lyra|Movement")
	bool bIsFalling;

protected:
	// 2. 重载CreateProxy和DestroyProxy来创建和销毁我们的代理实例
	virtual FAnimInstanceProxy* CreateProxy() override;
	virtual void DestroyProxy(FAnimInstanceProxy* InProxy) override;

	// NativeUpdateAnimation现在非常轻量,主要用于数据同步
	virtual void NativeUpdateAnimation(float DeltaSeconds) override;

	virtual void NativeInitializeAnimation() override;

protected:
	UPROPERTY(EditDefaultsOnly, BlueprintReadOnly, Category = "Lyra|Global Variables")
	TWeakObjectPtr<ALyraCharacter> LyraCharacter;
	UPROPERTY(EditDefaultsOnly, BlueprintReadOnly, Category = "Lyra|Global Variables")
	TWeakObjectPtr<UCharacterMovementComponent> CharacterMovement;

private:
	// 动画实例的代理。这是UAnimInstance和其Proxy之间的桥梁。
	UPROPERTY(Transient)
	FLyraBaseAnimInstanceProxy Proxy;
};
\end{amzcode}

\subsubsection{源文件 (LyraBaseAnimInstance.cpp)}

在源文件中,我们实现各个函数的具体逻辑。

*   PreUpdate (游戏线程): 从角色和移动组件获取数据,并将其存储在代理的成员变量中。
*   Update (工作线程): 使用代理中已存储的数据进行计算。**绝不**在此处访问任何UObject(如LyraCharacter)。
*   NativeUpdateAnimation (游戏线程): 将代理计算完成的结果拷贝回动画实例的属性,以便动画蓝图使用。

\begin{amzcode}{c++}
// LyraBaseAnimInstance.cpp

#include "AnimInst/LyraBaseAnimInstance.h"
#include "GameFramework/Character.h"
#include "GameFramework/CharacterMovementComponent.h"

// --- Proxy Implementation ---

FLyraBaseAnimInstanceProxy::FLyraBaseAnimInstanceProxy(UAnimInstance* Instance)
	: FAnimInstanceProxy(Instance)
{
}

void FLyraBaseAnimInstanceProxy::PreUpdate(UAnimInstance* InAnimInstance, float DeltaSeconds)
{
	// 在游戏线程执行:从动画实例和其拥有的对象(Actor等)安全地拷贝数据
	Super::PreUpdate(InAnimInstance, DeltaSeconds);

	ULyraBaseAnimInstance* LyraAnimInstance = Cast<ULyraBaseAnimInstance>(InAnimInstance);
	if (LyraAnimInstance && LyraAnimInstance->CharacterMovement.IsValid())
	{
		// 将需要的数据从游戏线程对象拷贝到代理的成员变量中
		CharacterVelocity = LyraAnimInstance->CharacterMovement->Velocity;
		bIsFalling = LyraAnimInstance->CharacterMovement->IsFalling();
		// ... 在这里拷贝所有其他需要的数据
	}
}

void FLyraBaseAnimInstanceProxy::Update(float DeltaSeconds)
{
	// 在工作线程执行:使用拷贝过来的数据进行所有耗时计算
	// 注意:这里绝对不能访问任何UObject指针(如LyraCharacter或CharacterMovement)
	Super::Update(DeltaSeconds);

	// 计算水平速度
	GroundSpeed = FVector(CharacterVelocity.X, CharacterVelocity.Y, 0.0f).Size();

	// ... 在这里执行所有其他的复杂计算
}


// --- AnimInstance Implementation ---

ULyraBaseAnimInstance::ULyraBaseAnimInstance()
{
	// 确保启用了多线程更新。如果动画蓝图的设置中未勾选,这行代码会强制开启。
	bUseMultiThreadedAnimationUpdate = true;
}

void ULyraBaseAnimInstance::NativeInitializeAnimation()
{
	Super::NativeInitializeAnimation();
	if (AActor* OwningActor = GetOwningActor())
	{
		LyraCharacter = Cast<ALyraCharacter>(OwningActor);
		CharacterMovement = LyraCharacter.IsValid() ? LyraCharacter->GetCharacterMovement() : nullptr;
	}
}

FAnimInstanceProxy* ULyraBaseAnimInstance::CreateProxy()
{
	// 创建代理实例
	return new FLyraBaseAnimInstanceProxy(this);
}

void ULyraBaseAnimInstance::DestroyProxy(FAnimInstanceProxy* InProxy)
{
	// 销毁代理实例
	delete InProxy;
}

void ULyraBaseAnimInstance::NativeUpdateAnimation(float DeltaSeconds)
{
	Super::NativeUpdateAnimation(DeltaSeconds);

	// 在游戏线程执行:从代理获取计算结果,更新到动画实例的属性上
	// 这样动画蓝图就可以使用这些最新的值
	GroundSpeed = Proxy.GetGroundSpeed();
	bIsFalling = Proxy.GetIsFalling();
}
\end{amzcode}

这个模式的核心思想是 **数据分离**:

1.  **游戏线程 (Game Thread)**: PreUpdate 负责从游戏世界收集原始数据(如速度、状态)。
2.  **工作线程 (Worker Thread)**: Update 接收这些拷贝好的、线程安全的数据,并执行所有复杂的计算。
3.  **游戏线程 (Game Thread)**: NativeUpdateAnimation (或 PostUpdate) 将计算结果同步回 UAnimInstance,供动画蓝图使用。

这种方式结构清晰,性能和扩展性都非常出色,完全能够满足您未来大量计算的需求。

\subsection{结论}

考虑到您“预计未来会有大量的求值计算”,强烈建议从一开始就使用 \il{FAnimInstanceProxy} 模式。

虽然初始设置稍微复杂,但它提供的清晰结构、高性能和强大的扩展性,将为项目未来的维护和性能优化带来巨大好处,完全符合您的长远需求。只重载 \il{NativeThreadSafeUpdateAnimation} 是一种捷径,但对于复杂系统而言,它很快会成为技术债务。
\end{document}
