\documentclass[10.5pt,a4paper]{ZBook}

\usepackage{menukeys}
\usepackage[normalem]{ulem}
\usepackage{indentfirst}
\setlength{\parskip}{.5ex}
\begin{document}
%\frontmatter
%\chapter*{前言}
%下一步定制section和目录。
%\tableofcontents
\mainmatter

\chapter{过程性动画}
Control rig 使用\cppsign{}实现步骤:
\section{球形追踪}
在程序化动画中,角色的脚部会直接移动到其原始动画位置,这会导致脚部穿过地板或其他障碍物,看起来不真实。

\subsection{解决方案:使用球体追踪(Sphere Trace)}
为了解决脚部穿透问题,可以使用\textbf{球体追踪}的方法来动态调整脚部的位置,使其能够正确地与地面或障碍物进行交互。

\subsubsection{实现步骤}
\begin{myenum}
    \item \textbf{创建球体追踪节点}:在动画蓝图中,使用“Sphere Trace by Channel”节点。这个节点会从一个起始点向一个结束点发射一个带半径的球体,检测路径上是否与其他物体发生碰撞。

    \item \textbf{设置追踪的起点和终点}:
    \begin{itemize}
        \item \textbf{终点 (End Point)}:设置为脚部IK的原始目标位置(即动画中脚应该在的位置)。
        \item \textbf{起点 (Start Point)}:在终点位置的基础上,增加Z轴的高度(例如50个单位),这样追踪射线就是从脚的上方垂直向下发射。
    \end{itemize}

    \item \textbf{处理追踪结果}:
    \begin{itemize}
        \item 球体追踪会返回一个布尔值(是否击中)和一个“击中位置”(\ci{Hit Location})。
        \item 如果追踪击中了物体(如地面),则将脚的IK目标位置设置为这个“击中位置”。
    \end{itemize}
\end{myenum}

\subsection{遇到的问题与优化}
\begin{enumerate}
    \item \textbf{问题一:脚在空中时位置错误}
    \begin{itemize}
        \item \textbf{现象}:当角色跳起,脚在空中时,球体追踪没有击中任何物体。此时,“击中位置”会返回一个零向量(0,0,0),导致双脚在空中时瞬间吸附到世界坐标原点。
        \item \textbf{解决方案}:使用一个\textbf{IF条件判断节点}。
        \begin{itemize}
            \item 如果球体追踪的返回值为\textbf{True}(击中物体),则使用“击中位置”作为脚的IK目标。
            \item 如果返回值为\textbf{False}(未击中),则继续使用原始的动画目标位置。
        \end{itemize}
    \end{itemize}

    \item \textbf{问题二:脚部陷入地面}
    \begin{itemize}
        \item \textbf{现象}:即使追踪击中了地面,脚看起来仍然部分陷入地面。这是因为骨骼的轴心点(Pivot)在脚踝,而不是在脚底。将脚踝位置设置为地面高度,自然会导致脚掌部分在地面以下。
        \item \textbf{解决方案}:在“击中位置”的基础上,给Z轴增加一个固定的偏移量(例如15个单位),然后再将其设置为IK目标。这个偏移量相当于从脚踝到脚底的高度,从而将脚正确地抬高到地面上。
    \end{itemize}

    \item \textbf{问题三:状态切换时的抖动}
    \begin{itemize}
        \item \textbf{现象}:当脚从空中接触地面时,由于Z轴偏移只在“击中”状态下应用,可能会产生一个微小的“吸附”或“抖动”效果。
        \item \textbf{解决方案}:将Z轴的高度偏移应用到\textbf{所有情况}下(无论是否击中物体)。这样可以确保脚部在空中和在地面的高度基准一致,消除状态切换时的不连贯。
    \end{itemize}
\end{enumerate}

\subsection{测试与结论}
视频中通过在场景中添加一个立方体并设置其碰撞属性,来测试脚部IK是否能正确地踩在新的障碍物上。实验证明该方法有效,可以实现单脚或双脚与不同高度的物体进行正确的交互。

然而,对于视频作者后续的纯程序化动画(不依赖传统动画文件)目标而言,这个基于追踪的IK修正系统并非必需。因此,在演示了其原理和实现方法后,作者最终删除了这部分逻辑,恢复到更简单的IK设置。

\end{document}